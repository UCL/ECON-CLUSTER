\documentclass{beamer}
\usepackage{hyperref}
\usepackage{amsmath}
\usepackage{listings}
\usetheme{Madrid}
\usecolortheme{beaver}
% \usepackage{beamerthemesplit} // Activate for custom appearance

\title{Using the Computer Science Cluster}
\author{Alan Crawford \\
           Lars Nesheim}
\date{June 27, 2018}

\begin{document}
\lstset{language=Pascal}
\frame{\titlepage}


\begin{frame}
\frametitle{Computer Science (CS) High-Performance Computing (HPC) Cluster}

\begin{itemize}
\item CS cluster has over 5200 nodes.
\item Designed to run large scale computing jobs in \textbf{batch mode} (non-interactive mode).
\begin{itemize}
\item \textbf{Limited graphics}-based interactive computing services. 
\end{itemize}
\item Users should set up their code to run in batch mode.
\item Users who primarily need graphics-based interactive computing should use Economics Department cluster, ISD services, or desktop computers. 
\end{itemize}
\end{frame}

\begin{frame}
\frametitle{More information}
\begin{itemize}
\item CS cluster webpage: \textcolor{blue}{\url{http://hpc.cs.ucl.ac.uk/how_to_login/}}.
\begin{itemize}
\item Get username and password from IT support,
\end{itemize}
\item Econ GitHub page: \textcolor{blue}{\url{https://github.com/UCL/ECON-CLUSTER}}
\begin{itemize}
\item Contact Fatima or Andrew to be added to the UCL user group.
\end{itemize}
\item If you need help, 
\begin{itemize}
\item CS support: \textcolor{blue}{\url{cluster-support@cs.ucl.ac.uk}}
\item ECON IT support:  \textcolor{blue}{\url{economics.it@ucl.ac.uk}}
\end{itemize}
\end{itemize}
\end{frame}

\begin{frame}
\frametitle{Obtain account}
\begin{itemize}
\item Fill out and submit application form (see Fatima or Andrew).
\item Wait 5-7 days and go to CS Cluster office: (4.20 Malet Place Engineering Building).
\item They will set up two accounts
\begin{enumerate}
\item CS departmental account (for remote access).
\item CS Cluster account (to use cluster).
\end{enumerate}
\item By default, both accounts have same username and password.
\end{itemize}
\end{frame}

\begin{frame}
\frametitle{Connect to cluster}

There are three ways to connect to the cluster:
\begin{enumerate}
\item Connect using ThinLinc (graphical interface).
\item Connect using ssh (command line interface).
\item Connect using ftp (data transfer).
\end{enumerate}
\end{frame}

\begin{frame}
\frametitle{Connect using ThinLinc}

Connecting using ThinLinc is a two-step process:
\begin{enumerate}
\item Connect to a \textbf{CSRW} (Computer Science Remote Worker) using your \textbf{CS Department Account} username and password.
\begin{itemize}
\item \textcolor{blue}{\href{http://www.cs.ucl.ac.uk/index.php?id=7404}{Instructions to dowload Thinlinc}}.
\end{itemize}
\item Then connect to one of the cluster login nodes (``vic" or ``wise") using your \textbf{CS Cluster Account} username and password.
\begin{enumerate}
\item Open terminal inside CSRW window. 
\item Then type one of the following
\begin{itemize}
\item ssh vic
\item ssh uctpXXX@vic
\end{itemize}
\item Enter your cluster password.
\end{enumerate}
\end{enumerate}
\end{frame}

\begin{frame}
\frametitle{Connect using ssh}
\begin{itemize}
\item For remote access (i.e. when not connected to UCL network), connecting is a two step process:
\begin{enumerate}
\item Connect to ``tails", ``storm" or ``jet" using your \textbf{CS Department Account} username and password.
\item Connect to the cluster using your \textbf{CS Cluster Account} username and password.
\end{enumerate}
\item For access from within the UCL network (e.g. a desktop in Drayon House or the Econ HPC), skip step 1.
\end{itemize}
\end{frame}

\begin{frame}
\frametitle{Step 1: connect to ``tails", ``jet" or ``storm"}
\begin{itemize}
\item Open an ssh client (e.g. putty) or a terminal.
\item Type
\begin{itemize}
\item ssh -X uctpXXX@tails.cs.ucl.ac.uk
\end{itemize} 
\item Enter your CS dept. password.
\item Note:
\begin{itemize}
\item The option `-X` allows graphics to be forwarded from the CS cluster to your computer.
\item The username ``uctpXXX" is that given when you are assigned your CS department account.
\item The text after the `@` is the address of the server.
\end{itemize}
\end{itemize}
\end{frame}

\begin{frame}
\frametitle{Step 2: connect to cluster}
\begin{enumerate}
\item Logon to the CS cluster using ssh.
\item Type 
\begin{itemize}
\item ssh -X uctpXXX@vic
\end{itemize}
\item Enter your CS cluster password.
\end{enumerate}
\end{frame}

\begin{frame}
\frametitle{Connect to the CS disk storage using sftp}
\begin{itemize}
\item To transfer files to cluster storage use sftp instead of ssh.
\begin{enumerate}
\item Open a terminal or an FTP client.
\item Type
\begin{itemize}
\item sftp uctpXXX@tails.cs.ucl.ac.uk
\end{itemize}
\item Enter your CS department password.
\end{enumerate}
\item From tails, you can directly access the CS data storage. 
\item For example, if you have previously  requested CS to 
create a storage directory named ``Nesheim-IO", you can access this by typing:
\begin{itemize}
\item cd /slash/economics/research/Nesheim-IO
\end{itemize}
\item See below requesting CS data storage.
\end{itemize}
\end{frame}

\begin{frame}
\frametitle{Typical workflow}
\begin{enumerate}
\item Logon to cluster.
\item Transfer data and/or files to cluster (using ftp or email).
\item Edit files or code.
\item Write a script to submit your job to SGE (Sun Grid Engine).
\begin{itemize}
\item SGE is the scheduler that manages allocation of jobs to nodes on the cluster.
\end{itemize}
\item Submit your job.
\item Monitor job progress if necessary. 
\item Download results to your local computer (using ftp or email).
\end{enumerate}

\end{frame}

\begin{frame}
\frametitle{SGE sessions: two types}
\begin{itemize}
\item There are two types of SGE sessions
\begin{enumerate}
\item Interactive sessions
\begin{itemize}
\item Start session with ``qrsh".
\item Interact with compute node directly on command line.
\item Good for developing code and testing things.
\end{itemize}
\item Non-interactive sessions
\begin{itemize}
\item Start session with ``qsub".
\item Job runs in background.
\item Good for running large scale, long running, or multiple jobs.
\item Cluster is optimised for this type of job.
\end{itemize}
\end{enumerate}
\item SGE jobs are command line only jobs.
\item If you need to do GUI based interactive work, it is best to use an alternative cluster. 
\end{itemize}
\end{frame}

\begin{frame}
\frametitle{Interactive SGE sessions}

\begin{itemize}
\item Use the ``qrsh" command and specify \textbf{running time} and \textbf{memory}.
\item For example, at the command line, type:
\begin{itemize}
\item \textbf{qrsh -l h\_vmem=1.9G,tmem=1.9G,h\_rt=8:0:0}
\end{itemize}
\item This command starts the session and requests 1.9 GB memory and 8 hours running time.
\begin{itemize}
\item `qrsh` is the login command for an interactive session.
\item `-l` is a flag for resource requests for the interactive session.
\end{itemize}
\item Resource options listed after the `-l` flag are:
\begin{enumerate}
\item `h\_vmem=XG,tmem=XG` requests X Gb of memory 
\item `h\_rt= H:M:S` requests that the session run for `H` hours, `M` minutes, `S` seconds
\end{enumerate}
\item For further command line options for `qrsh` type: `man qrsh`
\end{itemize}
\end{frame}

\begin{frame}
\frametitle{User tips}
\begin{itemize}
\item After starting SGE session, you need to load and open the software you require. See below for details.
\item It may take a short while (1 - 5 minutes) to be allocated a node.
\item To increase chances for quick allocation, request as little memory as necessary,
\item For a small job, 2G is likely to be sufficient. For Matlab, request at least 4G. 
\item If you need a lot of memory (i.e. X $>$ 2G), omit the `h\_rt` option from your `qrsh` command. 
\item For example, to request a 14G session type:
\begin{itemize}
\item qrsh -l h\_vmem=14G,tmem=14G
\end{itemize}
\end{itemize}

\end{frame}


\begin{frame}[fragile]
\frametitle{Submitting batch jobs}
To run a batch job:
\begin{itemize}
\item First, write a script (e.g.  a text file names `job1.sh`) detailing what resources to request from SGE and what commands/programs to run.
\item Then, submit the script using the command `qsub`.
\item For example, first create the file `job1.sh' and then type the command
\begin{itemize}
\item qsub job1.sh
\end{itemize}
\item Instructions for writing a script: \href{https://www.econ.ucl.ac.uk/wiki/index.php/Non-interactive_sessions}{job script instructions}.
\end{itemize}
\end{frame}

\begin{frame}
\frametitle{Some example command options to include in script}
\begin{itemize}
\item Request memory and running time:
\begin{semiverbatim}
\#\$ -l h\_rt=1:10:35
\end{semiverbatim}
\begin{semiverbatim}
\#\$ -l tmem=1.9G,h\_vmem=1.9G
\end{semiverbatim}
\item Each line containing SGE flags starts with `\#\$`.
\end{itemize}

\end{frame}

\begin{frame}
\frametitle{Submit batch job to compute node: non-interactive session}
\end{frame}

\begin{frame}
\frametitle{Job status}
\end{frame}

\begin{frame}
\frametitle{Disk storage}
\end{frame}

\begin{frame}
\frametitle{Software available}
\end{frame}

\begin{frame}
\frametitle{Memory management}
\end{frame}

\begin{frame}
\frametitle{Graphics on jake or elwood}

\end{frame}

\begin{frame}
\frametitle{Cluster rules}
\begin{itemize}
\item Never run jobs on head nodes (wise, vic, tails)
\item Store important data and/or large data on /SAN/economics/...
\item Avoid high frequency reading/writing of large datafiles to/from disk
\item If you need support, send email to cluster support, go visit them in person.
\item Update wiki page with tips for colleagues about how to solve common problems.
\item 
\end{itemize}
\end{frame}




\end{document}
